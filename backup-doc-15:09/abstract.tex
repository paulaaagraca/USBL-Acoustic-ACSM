\chapter*{Resumo}
%\addcontentsline{toc}{chapter}{Resumo}
Dispositivos robóticos programáveis como \textit{Autonomous Underwater Vehicles} (AUVs) são excelentes meios para exploração subaquática, já que são capazes de executar missões de longa duração com variadas possibilidades de aplicação e objetivos. Neste sentido, o conceito de mula AUV surgiu como mecanismo útil que periodicamente recolhe dados dos AUVs em missão. Para que tal seja possível, é necessário implementar um sistema de localização e posicionamento robusto que permite aos AUVs encontrarem outros veículos de forma a aproximarem-se deles eficientemente.

A presente dissertação foca-se na implementação de mecanismos que levam a um aumento de precisão na localização subaquática usando um sistema USBL (Ultra-Short Baseline), para curtas e longas distância. Primeiro, é descrito o design da arquitetura de um modulo capaz de melhorar a precisão da medida dos tempos de chegada de sinais enviados por uma fonte acústica. De seguida, é conduzido um estudo sobre possíveis métodos de avaliação do desempenho de uma configuração de sensores, já que consiste num aspecto crucial na precisão de estimação. Por último, o método de seleção adaptativa de configurações é apresentado, o qual serve como ferramenta que seleciona um conjunto de hidrophones, a partir de um grupo discreto em posições fixas, que leva a uma maior precisão na localização. Este método pretende retificar problemas que surgem em sistemas USBL clássicos. 

Após a implementação, todos os mecanismos desenvolvidos foram sujeitos a testes detalhados em simulação que validam o seu funcionamento e demonstram resultados satisfatórios em condições controladas. Adicionalmente, foram realizados testes no tanque do DEEC e em mar aberto para avaliar as melhorias alcançadas nas medidas dos tempos de chegada.

\chapter*{Abstract}
%\addcontentsline{toc}{chapter}{Abstract}
Robotic programmable devices such as Autonomous Underwater Vehicles (AUVs) are great means for underwater exploration, as they are capable of executing long term missions with many possible applications and goals. In this regard, the concept of mule AUVs arises as a valuable mechanism to periodically collect data from survey AUVs during the missions. In order to achieve this, a robust localization system needs to be implemented allowing the mule AUV to find the other vehicle and draw near it efficiently.

The present dissertation focuses on the implementation of mechanisms that lead to an increase in underwater localization precision using an USBL (Ultra-Short Baseline) system, for both short and long range. Firstly, it is described the architecture design of a module that is capable of improving the precision of the time of arrival measurement of signals sent by an acoustic transmitter. Then, a study is conducted on possible methods for evaluating a sensor configuration performance, as it consists on a crucial aspect in estimation precision. Lastly, the adaptive configuration selection method is presented, which serves as a tool that selects a set of hydrophones, from a discrete group in fixed positions, that leads to the highest localization precision. This method intends to rectify issues that arise from classic USBL systems.

After implementation, all developed mechanism were subjected to comprehensive simulated tests that validate its function and demonstrate successful results with controlled conditions. Additionally, tests were performed in DEEC's tank and in open sea to evaluate the achieved improvement on the time of arrival measurements.