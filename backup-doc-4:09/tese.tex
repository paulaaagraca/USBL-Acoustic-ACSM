%% FEUP THESIS STYLE for LaTeX2e
%% how to use feupteses (portuguese version)
%%
%% FEUP, JCL & JCF, 31 Jul 2012
%%
%% PLEASE send improvements to jlopes at fe.up.pt and to jcf at fe.up.pt
%%

%%========================================
%% Commands: pdflatex tese
%%           bibtex tese
%%           makeindex tese (only if creating an index)
%%           pdflatex tese
%% Alternative:
%%          latexmk -pdf tese.tex
%%========================================

\documentclass[11pt,a4paper,twoside,openright]{report}

%% For iso-8859-1 (latin1), comment next line and uncomment the second line
\usepackage[utf8]{inputenc}
%\usepackage[latin1]{inputenc}

%% Portuguese version

%% MIEIC options
%\usepackage[portugues,mieic]{feupteses}
%\usepackage[portugues,mieic,juri]{feupteses}
%\usepackage[portugues,mieic,final]{feupteses}
%\usepackage[portugues,mieic,final,onpaper]{feupteses}

%% MIEEC options
%\usepackage[portugues,mieec]{feupteses}
%\usepackage[portugues,mieec,juri]{feupteses}
%\usepackage[portugues,mieec,final]{feupteses}
%\usepackage[portugues,mieec,final,onpaper]{feupteses}

%% For other degrees
\usepackage[portugues]{feupteses} % you must define the degree bellow

%% Options: 
%% - portugues: titles, etc in portuguese
%% - onpaper: links are not shown (for paper versions)
%% - backrefs: include back references from bibliography to citation place

%% Uncomment to create an index (at the end of the document)
%\makeindex

%% Path to the figures directory
%% TIP: use folder ``figures'' to keep all your figures
\graphicspath{{figures/}}

%%----------------------------------------
%% TIP: if you want to define more macros, use an external file to keep them
%some macro definitions

% format
\newcommand{\class}[1]{{\normalfont\slshape #1\/}}

% entities
\newcommand{\Feup}{Faculdade de Engenharia da Universidade do Porto}

\newcommand{\svg}{\class{SVG}}
\newcommand{\scada}{\class{SCADA}}
\newcommand{\scadadms}{\class{SCADA/DMS}}

\usepackage{todonotes}
\newcommand{\note}[1]{\todo[inline]{#1}}
%%----------------------------------------

%%========================================
%% Start of document
%%========================================
\begin{document}

%%----------------------------------------
%% Information about the work
%%----------------------------------------
\title{Título da Dissertação}
\author{Nome do Autor}

%% Comment next line if not necessary for degree name
\degree{Programa Doutoral em Engenharia Informática}

%% Uncomment next line for date of submission
%\thesisdate{31 de julho de 2008}

%% Comment next line for copyright text if not used
\copyrightnotice{Nome do Autor, 2008}

\supervisor{Orientador}{Nome do Orientador}

%% Uncomment next line if necessary
%\supervisor{Co-orientador}{Nome de Outro Orientador}

%% Uncomment committee stuff in the final version if used
%\committeetext{Aprovado em provas públicas pelo Júri:}
%\committeemember{Presidente}{Nome do presidente do júri}
%\committeemember{Arguente}{Nome do arguente do júri}
%\committeemember{Vogal}{Nome do vogal do júri}

%% Uncomment signature line in the final on paper version if used
%\signature

%% Specify cover logo (in folder ``figures'')
\logo{uporto-feup.pdf}
 
%% Uncomment next line for additional text below the author's name (front page)
%\additionalfronttext{Preparação da Dissertação}

%%----------------------------------------
%% Preliminary materials
%%----------------------------------------

% remove unnecessary \include{} commands
\begin{Prolog}
  \chapter*{Resumo}
%\addcontentsline{toc}{chapter}{Resumo}
Dispositivos robóticos programáveis como \textit{Autonomous Underwater Vehicles} (AUVs) são excelentes meios para exploração subaquática, já que são capazes de executar missões de longa duração com variadas possibilidades de aplicação e objetivos. Neste sentido, o conceito de mula AUV surgiu como mecanismo útil que periodicamente recolhe dados dos AUVs em missão. Para que tal seja possível, é necessário implementar um sistema de localização e posicionamento robusto que permite aos AUVs encontrarem outros veículos de forma a aproximarem-se deles eficientemente.

A presente dissertação foca-se na implementação de mecanismos que levam a um aumento de precisão na localização subaquática usando um sistema USBL (Ultra-Short Baseline), para curtas e longas distância. Primeiro, é descrito o design da arquitetura de um modulo capaz de melhorar a precisão da medida dos tempos de chegada de sinais enviados por uma fonte acústica. De seguida, é conduzido um estudo sobre possíveis métodos de avaliação do desempenho de uma configuração de sensores, já que consiste num aspecto crucial na precisão de estimação. Por último, o método de seleção adaptativa de configurações é apresentado, o qual serve como ferramenta que seleciona um conjunto de hidrophones, a partir de um grupo discreto em posições fixas, que leva a uma maior precisão na localização. Este método pretende retificar problemas que surgem em sistemas USBL clássicos. 

Após a implementação, todos os mecanismos desenvolvidos foram sujeitos a testes detalhados em simulação que validam o seu funcionamento e demonstram resultados satisfatórios em condições controladas. Adicionalmente, foram realizados testes no tanque do DEEC e em mar aberto para avaliar as melhorias alcançadas nas medidas dos tempos de chegada.

\chapter*{Abstract}
%\addcontentsline{toc}{chapter}{Abstract}
Robotic programmable devices such as Autonomous Underwater Vehicles (AUVs) are great means for underwater exploration, as they are capable of executing long term missions with many possible applications and goals. In this regard, the concept of mule AUVs arises as a valuable mechanism to periodically collect data from survey AUVs during the missions. In order to achieve this, a robust localization system needs to be implemented allowing the mule AUV to find the other vehicle and draw near it efficiently.

The present dissertation focuses on the implementation of mechanisms that lead to an increase in underwater localization precision using an USBL (Ultra-Short Baseline) system, for both short and long range. Firstly, it is described the architecture design of a module that is capable of improving the precision of the time of arrival measurement of signals sent by an acoustic transmitter. Then, a study is conducted on possible methods for evaluating a sensor configuration performance, as it consists on a crucial aspect in estimation precision. Lastly, the adaptive configuration selection method is presented, which serves as a tool that selects a set of hydrophones, from a discrete group in fixed positions, that leads to the highest localization precision. This method intends to rectify issues that arise from classic USBL systems.

After implementation, all developed mechanism were subjected to comprehensive simulated tests that validate its function and demonstrate successful results with controlled conditions. Additionally, tests were performed in DEEC's tank and in open sea to evaluate the achieved improvement on the time of arrival measurements. % the abstract
  \chapter*{Agradecimentos}
%\addcontentsline{toc}{chapter}{Agradecimentos}
\noindent
Aos meus orientadores: professor José Carlos Alves, pelo incansável apoio, inspiração e motivação desde o primeiro ano de faculdade, e Bruno Miguel Ferreira, pela disponibilidade, interesse e constante incentivo.

\noindent
À FEUP, por testar o meu limite em todos os sentidos, e ao projeto GROW desenvolvido pelo CRAS (INESC TEC), por me abrir portas a desafios interessantes num tema que me relembrou a razão pela qual quis seguir engenharia.

\noindent
Aos amigos da faculdade e do IEEE UP SB, por me guiarem numa realidade que me vai sendo cada vez menos desconhecida, por me inspirarem a estabelecer objetivos ambiciosos para mim própria, pelas longas noites, pela melhor experiência de sempre em Munique.

\noindent
Aos amigos de longa data, por me relembrarem das minhas raízes e de quem sou, pelas conversas intermináveis, pelos abracinhos e pelo apoio incondicional.

\noindent
Ao meu namorado, por todas as experiências, pela paciência interminável e por simplesmente acreditar.

\noindent
Aos meus pais e irmã, pelos miminhos e por tornarem tudo isto possível.

\vspace{10mm}
\flushleft{Paula}
\\ 
Setembro 2020
  % the acknowledgments
  \cleardoublepage
\thispagestyle{plain}

\vspace*{8cm}

\begin{flushright}
   \textsl{``A curiosidade leva por um lado a escutar às portas \\
   					e por outro a descobrir a América''} \\
\vspace*{1.5cm}
           Eça de Queirós
\end{flushright}
    % initial quotation if desired
  \cleardoublepage
  \pdfbookmark[0]{Conteúdo}{contents}
  \tableofcontents
  \cleardoublepage
  \pdfbookmark[0]{Lista de Figuras}{figures}
  \listoffigures
  \cleardoublepage
  \pdfbookmark[0]{Lista de Tabelas}{tables}
  \listoftables
  \chapter*{Abbreviations}
%\addcontentsline{toc}{chapter}{Abbreviations}
\chaptermark{ABBREVIATIONS}

\begin{flushleft}
\begin{tabular}{l p{0.8\linewidth}}
AUV		& Autonomous Underwater Vehicle \\
BPSK    & Binary Phase Shift Keying \\
CC 		  & Cross-Correlation \\
DEEC	& Departamento de Engenharia Electrotécnica e de Computadores \\
FIM		  & Fisher Information Matrix \\
FPGA    & Field-Programmable Gate Array \\
FSK		  & Frequency-Shift Keying \\
GCC      & Generalized Cross-Correlation \\
HDL		 & Hardware Description Language\\
LBL		  & Long Baseline\\
LOS		 & Line-of-sight\\
MF		  & Medium Frequency \\
ML		  & Maximum Likelihood \\
RMS		 & Root Mean Square \\
RSSI 	  & Received Signal Strength Indicator \\
SBL		  & Short Baseline \\
SNR		 & Signal-Noise Ratio\\
TDE 	 & Time Delay Estimation \\
TDOA   & Time Difference of Arrival \\
TOA	   	& Time of Arrival \\
TOF		 & Time of Flight \\
USBL    & Ultra-Short Baseline
\end{tabular}
\end{flushleft}  % the list of abbreviations used
\end{Prolog}

%%----------------------------------------
%% Body
%%----------------------------------------

\StartBody

%% TIP: use a separate file for each chapter
\chapter{Introduction} \label{chap:intro}

\section{Context and Motivation} \label{sec:context}

Today, the deep blue ocean still represents a relevant topic of research in the scientific community as it constantly rises new unexplained mysteries. Up to now, only 15\% of the entire ocean floor is mapped based on collected data \cite{deeperblue}. As such, it seems essential to create efficient research tools to improve the discovery of information.

%________________________________________

Robotic autonomous underwater vehicles (AUVs) are great means for diverse applications in underwater exploration using variable resource requirements and duration, such as monitoring structures installed in shallow waters or exploring the deep ocean floor for scientific purposes. Particularly in long-term missions, the AUV is usually deployed using a docking system and it navigates underwater until the end of the mission, when it returns to the base station. Thus far, the data that is being collected is typically not accessible by any processing system or researchers. 

A method that is used to resolve this limitation is employing additional mule AUVs, whose goal is to travel near the survey AUV, collect its data during the mission's term and return in a relatively short time period. This allows the data to be periodically processed during the mission, which facilitates the definition of future courses for the mission, such as shortening its duration or sending additional commands. In the mentioned localization system, high accuracy is key as it lowers the resource consumption, saves up time in the inherently slow global process and avoids missing the AUV's underwater localization.

The described process can only be achieved if the mule AUV is able to locate the other vehicle and draw near it. For that reason, a USBL (Ultra-Short Baseline) system is used to receive the transmitted signals and calculate the angle of arrival of the acoustic signal, thus the direction that the mule AUV should navigate. Additionally, using a synchronization mechanism, the mule is also able to determine the distance to the acoustic source and thus the vehicles' relative positions.

In such scenario, the USBL system needs to meet specific requirements to assure a reliable localization. Since the acoustic source can be located anywhere, it is essential that the estimation is accurate for both short and long range distances. Additionally, the system needs to have line of sight in any direction, which is compromised from the start by deploying the sensors on an opaque AUV. Typically, the available USBL commercial solutions have a limited range for these characteristics, so the development of such system constitutes a technological challenge.

Therefore, this dissertation intends to explore a method that assumes the deployment of multiple hydrophones in a vehicle, from which only four are used simultaneously for the angle of arrival estimation. This allows to dynamically optimize the used sensor configuration according to which returns the lowest estimation error, so it is possible to always have line of sight to the target, independently of navigation maneuvers, and improving the estimation accuracy. Overall, this mechanism intends to rectify issues that arise from classic USBL systems, such as the before mentioned. 

In the course of this document, the dynamically reconfigurable configuration method is detailed and refined, determining its limitations and capabilities. Additionally, all the contemplated tools and complementary modules are carefully explained.

This research work falls under the scope of activities developed by the Center of Robotics and Autonomous Systems of INESC TEC. It is integrated in the GROW project which focuses on exploring the use of AUVs as data mules for long duration missions.


\section{Objectives} \label{sec:objective}

The goal of the present work is to study and propose a dynamically reconfigurable configuration method, which assumes the integration of several hydrophones in a USBL system to allow selecting the set of sensors that minimizes the estimation error. This aims to achieve high estimate accuracy for both short and long range distances and provide full line of sight from the chosen hydrophones to the target that can be located anywhere. In order to attain this, a comparative study is developed on tools that allow to compare the performance of sensors configurations in order to select the most reliable option. Then, the proposed system is presented in detail and validated with comprehensive simulations.

In addition to the main objective, it is intended to implement and validate a digital signal processing system for FPGA technology, which calculates the difference between the times of arrival of an encoded acoustic signal to four hydrophones.  Thereafter, a estimator is developed which is be able to take the time differences of arrival in order to estimate the intended angle of arrival.

\section{Document Structure}

The present document is partitioned into six chapters, which are summarized in this section.

Chapter \ref{chap:sota} offers an overview on background concepts about underwater acoustics, localization estimation and positioning systems, followed by USBL available commercial solutions and developed technology for a similar purpose. Then it focuses on angle of arrival determination methods and optimization mechanisms that are typically employed.

After getting in touch with the terms and reviewing the literature, chapter \ref{chap:problem} intends to clarify the problem that is being resolved in this thesis. The research hypothesis is stated as well as the research questions that are discussed and indented to be further explored. The chapter ends with the clarification of the used validation methods for the work. 

Chapter \ref{chap:proposed_sys} presents and explains the developed hardware design for the phase difference calculation. Then, three different approaches are presented for systematic comparison between the performance of a sensor configuration. These are supported with simulation experiments which allow to draw conclusions on the preferred approach.

Chapter \ref{chap:study} details the developed dynamically reconfigurable configuration method. The theoretical specifics and thought process are laid out and the mechanism is then validated through simulations.

Lastly, chapter \ref{chap:conclusion} gives the final remarks about the developed work and mentions research work which could be further developed in the future.  
 
\chapter{State of the Art} \label{chap:sota}

This chapter presents the fundamental concepts of underwater acoustics engineering for localization and positioning of aquatic autonomous vehicles.

\section{Underwater acoustic channel} \label{sec:acoustchann}

Although satellite based navigation systems are the most commonly used for positioning and localization at the air, the used radio signals are highly absorbed by the water and thus inappropriate for underwater localization and also for communications. Therefore, the state of the art solutions for long range localization and communications rely on the propagation of acoustic signals.

The natural limitations of acoustic channels combined with the properties of an underwater environment, result in challenges and limitations in developing communication and localization systems \cite{survey-tech-chall}:
\begin{itemize}
	\item Long propagation delays;
	\item Variable speed of the acoustic signals;
	\item Reference nodes may have different drifting rates from each other due to water currents, which leads to uncertainties on the definition of absolute times and synchronization;
	\item Limited bandwith
	\item Signals are bended due to sound speed variation along the water column and shadowed in many different surfaces, which may lead to the incorrect detection of the line-of-sight (LOS) signal;
	\item Attenuation and asymmetric signal-to-noise ratio, which arises from SNR depending on depth and frequency with complex behaviors that depend on the characteristics of the environment;
\end{itemize}

\subsection{Speed of sound} \label{subsec: speed-sound}

The oceanic environment has a complex sound propagation model, as it comprises many variants in order to realistically represent underwater acoustics.

Acoustic signals' propagation speed is mainly related with two factors: compressibility and density. The water density can be characterized by the temperature, salinity and pressure,which is associated with depth. Figure \ref{fig:spd-sound} exhibits a generic sound speed profile in relation to depth. The water surface is commonly a mixed layer which results in an approximately constant sound speed. After this layer, it suffers a significative decrease, usually reaching the lower tangible speed, which results from the variation of temperature that characterizes the thermocline layer. From that point forward, pressure is the greatest influencer on the speed of sound, so it increases relatively proportionally to depth .

\begin{figure}[!htbp]
	\centering
	\includegraphics[width=0.6\textwidth]{figures/sound-profile}
	\caption{Generic sound speed profile}
	\label{fig:spd-sound}
\end{figure}

The empirical equation \ref{eq:spd-sound} \cite{ocean-acoust} is a simplified translation of the behavior of the sound speed \textit{c} in meters per second, with relation to the temperature \textit{T} in ºC, the salinity \textit{S} in parts per thousand and the depth \textit{z} in meters. 
\begin{eqnarray}
c = 1449.2 + 4.6T - 0.055T^2 + 0.00029T^3 + (1.34 - 0.01T)(S-35) + 0.016z 
\label{eq:spd-sound}
\end{eqnarray}


\subsection{Multipath} \label{subsec:multipath}

Multipath occurs when a transmitting signal suffers reflection or refraction in a surface (e.g. water surface, ocean floor, dock's wall), leading to a change in its original characteristics. This phenomenon can affect the propagation speed, the energy and the total distance that the signal was predicted to travel. These altered signals in conjunction with constant movement of the receiver makes it more complicated to accurately estimate the distance between the transmitter and the receiver, as well as determine the line-of-sight signal. 

\begin{figure}[!htbp]
	\centering
	\includegraphics[width=0.7\textwidth]{figures/multipath}
	\caption{Multipath}
	\label{fig:mpath}
\end{figure}

In consequence, the underwater acoustic channel is qualified as a non-minimum phase system because it produces time-variant output responses.

\subsection{Doppler Effect} \label{subsec:doppler}

In a communication and localization system between two entities moving with non-zero relative velocity, if a transmitter sends a signal with a certain operation frequency to the receiver, then the perceived frequency by the receiver will suffer a shift from the original signal. This frequency difference is expressed as a Doppler shift and explained by the Doppler Effect.

The magnitude of the generated frequency shift can be expressed as a ratio \ref{eq:ratio}, where the transmitter-receiver velocity is compared to \(c\), the speed of sound \cite{commchan}.

\begin{eqnarray}
&a = \frac{v}{c}
\label{eq:ratio}
\end{eqnarray}

Autonomous Underwater Vehicles (AUVs) usually move with velocities in the order of few meters per second. Therefore, the \(a\) factor mentioned above has a significant value and needs to be considered when implementing synchronization systems, as well as developing estimation algorithms.

In certain localization and communication systems, it is critical to correct the Doppler effect because data can be compromised (e.g. FSK modulated signals, in which information is codified into frequency changes). A simple Doppler compensation process  was proposed in \cite{thesis-joao}, in a system to detect phase-modulted binary sequences using cross-correlation.

This phenomenon can also be explored to determine the relative velocity between two devices, by measuring the frequency deviation with respect to the frequency expected to be received.


\subsection{Attenuation and signal-to-noise ratio} \label{subsec:snr}

When considering underwater communication systems, it is essential to quantify the attenuation of the channel, i.e. the part of the signal's energy which is absorbed by the involving surrounding. In underwater channels, this absorbance is frequency variable and it is also dependent on physical characteristics of the water, as salinity and temperature. 

The underwater acoustic channel has a particular model that describes its attenuation path loss \(A(d,f)\), given in logarithmic scale by equation \ref{eq:attenuationdb} \cite{pathloss}. 
\begin{eqnarray}
&10\ log(A(d,f)) = 10\ k\ log(d) + d\ 10\ log(a(f))
\label{eq:attenuationdb}
\end{eqnarray}
From the equation, \(d\) is the distance from the transmitter to the receiver in kilometers (Km), \(f\) is the operating frequency in kilohertz (KHz), \(10klog(d)\) represents the spreading loss which describes how the sound level (in decibel, dB) decreases as the sound wave spreads, \(d10log(a(f))\) is the absorption loss that a signal suffers during its propagation path, \(k\) is the spreading factor which is related with the considered configuration (e.g. cylindrical, spheric, etc.), \(a(f)\) is the absorption coefficient that can be obtained through the equation in \cite{pathloss}.

Noise is another factor that is considered when analyzing a real underwater acoustic channel, as it defines the signal-to-noise ratio (SNR) that characterizes the channel. The SNR is dependent on the attenuation level which increases with frequency, and the noise which decays with frequency. Consequently, the SNR varies over the signal bandwidth and it is asymmetric. The equation \ref{eq:snr} \cite{commchan} expresses this relationship, where \(S_{d}(f)\) represents the power spectral density of the transmitted signal.
\begin{eqnarray}
&SNR(d,f) = \frac{S_{d}(f)}{(A(d,f))\ N(f)}
\label{eq:snr}
\end{eqnarray}

%________________________________________________________

\section{Range estimation for underwater localization}

Underwater localization takes into consideration the distance between the target object to track and the reference point. As consequence, it is always relevant to apply methods which easily and effectively determine this range.

There are two main types of techniques that are used to achieve such objective: the Received Signal Strength Indicator (RSSI) and the Time Delay Estimation (TDE).


\subsection{Received Signal Strength Indicator}

The Received Signal Strength Indicator (RSSI) method is based on the strength of the signal that reaches the target. It determines the distance between the target and the reference node by analyzing the received signal strength and comparing it with an underwater attenuation model which is range dependent [\hyperref[r:ocean-acoust]{3}]. 

Since the underwater acoustic channel suffers from multipath, time variance and high overall path loss, the RSSI technique is not adequate for underwater applications.

\subsection{Time Delay Estimation}
Time Delay Estimation (TDE) mechanisms use a pair of nodes, the target and the reference point, to measure the range between them. This distance is based on the time that it takes for a signal to travel from the reference point to the target.

There are three main categories that devide TDE methods, which are Time Difference of Arrival (TDOA), Time of Arrival (TOA) and Time of Flight (TOF).

\subsubsection{Time of Arrival}

Time of Arrival (TOA) is interpreted as the time delay between the transmission of a signal in the reference node until its reception on the target node. Although this is the conceptually simplest method to employ, it requires synchronization between the nodes since the target entity needs to know the instance when the signal was sent to be able to calculate the difference.

Considering a generic transmitted signal \textit{s(t)}, the received signal can be expressed as \ref{eq:toa}, where $\tau$ represents the time of arrival and \textit{n(t)} is white noise with zero mean \cite{wirelesscomm}. 

\begin{eqnarray}
& r(t) = s(t - \tau) + n(t)
\label{eq:toa}
\end{eqnarray}

\subsubsection{Time Difference of Arrival}

The Time Difference of Arrival (TDOA) is a technique that compares the time of arrival of a signal to different hydrophones in order to estimate the angle of arrival of the acoustic signal. The array of reception hydrophones have known determined positions among them so that is is possible to compare the different times of arrival or phase differences. This method can be employed using a uni-directional signal or a round trip communication.

There are several algorithms and mathematical models that can be employed to execute the TDOA method, such as the Cross-Correlation and Maximum Likelihood.

\subsubsection{Generalized Cross-Correlation}

The Generalized Cross-Correlation (GCC) method is used to generically represent the relationship strength between two signals.

Considering two distanced hydrophones in the same environment and an acoustic source \textit{s(t)}, \textit{x1(t)} and \textit{x2(t)} are the signals received by each of the two hydrophones. The equations \ref{eq:gcc1} and \ref{eq:gcc2} \cite{crosscorr} express the mentioned signals in relation to \textit{w1(t)} and \textit{w2(t)} which are Gaussian noise coefficients uncorrelated with the source, $\tau$ that represent the delay and $\alpha$ which is an attenuation function.
\begin{eqnarray}
&x1(t) = s(t) + w1(t)
\label{eq:gcc1}\\
&x2(t) = \alpha s(t - \tau) + w2(t)
\label{eq:gcc2}\\
\end{eqnarray}

From these expressions, the generalized cross-correlation function between signals \textit{x1(t)} and \textit{x2(t)} is given by \ref{eq:gcc3}. The $G_{x1x2}(f)$ is the spectrum of the cross-correlation. The $\psi(f)$ represents a prefilter and it is essentially the distinctive parameter that originate various different methods of cross-correlation, since it should depend on different environments and properties as SNR. 
\begin{eqnarray}
& R_{x1x2}(\tau) = \int_{-\infty}^{\infty} \psi(f) G_{x1x2}(f)\ e^{i2\pi f\tau} df
\label{eq:gcc3}\\
& T = \tau_{max} [ R_{x1x2}(\tau) ]
\label{eq:gcc4}
\end{eqnarray}

Finally, the maximum value of $R_{x1x2}(\tau)$, expressed in \ref{eq:gcc4}, is the so called correlation peak and provides information about the time delay $\tau$ which is the main matter of Time Delay Estimation. 


\subsubsection{Cross-Correlation}

After approaching the generalized method of cross-correlation, it is possible to better understand the Cross-Correlation (CC) method. There are two main variations of CC \cite{crosscorr}, which are the slow cross-correlation in the  time domain and the fast cross-correlation in the frequency domain. The second approach is based on the Fast Fourier Transform as it locates the peak by analyzing frequency similarities between the signals. 

Th Cross-Correlation technique uses a prefilter $\psi(f)$ equal to 1, as it is the simplest method of its kind.

\subsubsection{Maximum Likelihood}

The Maximum Likelihood (ML) method is a variation of Cross-Correlation which uses the prefilter $\psi(f)$ represented mathematically by \ref{eq:ml1}, where $\gamma_{12}(f)$ is a function of spectrum of cross-correlation $G_{x1x2}(f)$ and spectrum of auto-correlations $G_{x1x1}(f)$, $G_{x2x2}(f)$ as expressed in \ref{eq:ml2} \cite{crosscorr}.

\begin{eqnarray}
& \psi(f) = \frac{|\gamma_{12}(f)|^2}{|G_{x1x2}(f)|[1-|\gamma_{12}(f)|^2]}
\label{eq:ml1} \\
& |\gamma_{12}(f)|^2 = \frac{|G_{x1x2}(f)|^2}{G_{x1x11}(f) . G_{x1x11}(f)]}
\label{eq:ml2} 
\end{eqnarray}

There is also a version of ML that uses the power spectral densities of the signals, which can be helpful for calculations in various applications. 

\subsubsection{Time of Flight}

Time of Flight (TOA) measures essentially the round-trip time communication between two nodes. The target node sends a signal to the reference node, which has an integrated transponder that responds transmitting a signal back to the target. T4he TOA is then estimated as the time interval from the moment the first signal is transmitted until the moment the second signal is received by the same node. 

This method may be used without additional synchronization systems as it assumes that the response signal is sent immediately after the received one and the intrinsic transmitting delays are known.

The accuracy of this technique depends mainly on the environment conditions, which include the water properties and the surrounding reflection surfaces which cause multipath. Therefore, the mechanism is susceptible of variable errors according to the location and characteristics of its employment.

%________________________________________________________

\section{Localization estimation}

In networks with multiple nodes is typical to use localization estimation to establish position relationships between elements. The operation principal is usually to have a set of reference nodes with known positions so that it is possible to determine the relative positions between each reference node and the target. 

An extensive comparison of different localization schemes for underwater sensors networks can be consulted in \cite{suvey-loc}.

\subsection{Triangulation}

Triangulation is a method of localization based on the measurement of angles which are related to the reference beacons and the target object. 

\subsubsection{Three-Object Triangulation}
The simplest method of this category is the Three-Object Triangulation, which considers a configuration as illustrated in figure \ref{fig:tri1}. It is assumed that the location of the beacons is pre-configured and the environment is obstacle-free. $\lambda_{12}$ is the angle formed by the intersection of the straight lines [O,1] and [O,2]. Similarly, $\lambda_{31}$ is the angle formed by the intersection of the straight lines [O,1] and [O,3]. Using these two sets of nodes, we can trace circumferences that include their coordinates and as a consequence their intersection will correspond to the location of the target.

\begin{figure}[!htbp]
	\centering
	\includegraphics[width=0.4\textwidth]{figures/triangulation}
	\caption{Three-Object Triangulation}
	\label{fig:tri1}
\end{figure}

Although this is a very straightforward technique to implement, it does not cover all possible scenarios, namely when the three beacons and the object are all placed in the same circumference or when the environment has obstacles between nodes.

\subsubsection{Geometric Triangulation algorithm}

A more complex method relies on the Geometric Triangulation algorithm. 

\begin{figure}[!htbp]
	\centering
	\includegraphics[width=0.5\textwidth]{figures/geotriang}
	\caption{Geometric Triangulation algorithm}
	\label{fig:tri2}
\end{figure}


Considering a Cartesian plane with defined lengths L1, L12 and L31, as shown in image \ref{fig:tri2}, it is possible to establish trigonometrical relationships that estimate the location of the object within the created triangular areas. The position of the target is given by coordinates (xT,yT) and can be calculated through equations \ref{eq:tri1} and \ref{eq:tri2}. (x1,y1) represents the location of beacon 1 and L1 is the distance between this beacon and the object. The trigonometric relationships for calculating the mentioned variables can be consulted in \cite{triangalgo}.

\begin{eqnarray}
& xT = x1 - L1 * cos(\phi + \tau)
\label{eq:tri1}\\
& yT = y1 - L1 * sin(\phi + \tau)
\label{eq:tri2}
\end{eqnarray}

\subsection{Trilateration}

Trilateration is a technique that does not rely on calculations using angles but instead it uses distances to locate an object.

Considering a scenario with three reference beacons, the distance between the target and each one of the beacons is taken as the radius of a circumference. By doing this, it is possible to obtain three circumferences that intersect each other. With only two circumferences, there are two possible locations for the object, however, when added the third circumference the exact location is obtained. The 2D coordinates are obtained by solving systems of equations with the circle equation \ref{eq:circle} \cite{trilat}, where $(x_{i}, y_{i})$ is the beacon coordinates and $r_{i}$ is the distance between the beacon and the object.

\begin{eqnarray}
& (x - x_{i})^2 + (y - y_{i})^2 = r_{i}^2 
\label{eq:circle}
\end{eqnarray}

Trilateration is commonly used in underwater acoustic localization, as it used to find a relative position of the target in two dimensions and additionally determines the depth as third dimension, by using a pressure sensor with high accuracy.

\subsection{Multilateration}

Multilateration is a generalization of the trilateration technique, as it uses the same conceptual principal with multiple reference beacons instead of exactly three. In this method, the employment of \textit{n+1} nodes will allow to determine \textit{n} coordinates \cite{arch_localiz}. For example, determining the position (x,y,z) of a target, would require to resolve a system of equations using \ref{eq:mult}. $(x_{i}, y_{i}, z_{i})$ is the coordinates of the beacon and $d_{i}$ is the distance between the beacon and the target.

\begin{eqnarray}
& (x - x_{i})^2 + (y - y_{i})^2 + (z - z_{i})^2 = d_{i}^2 
\label{eq:mult}
\end{eqnarray}

Distributed mechanisms, such as multilateration, are usually divided in three phases of positioning \cite{suvey-loc}:
\begin{itemize}
	\item Distance estimation between the reference nodes and target object, usually using TDOA or TOF mechanisms;
	\item Position estimation, usually obtained by solving a system of linear equations through mathematical efficient techniques;
	\item Final refinement of the measurement in order to improve accuracy.
\end{itemize}

As an alternative to solve localization issues using circumferences, multilateration can also take advantage of a hyperbola-based localization method. Considering a target at (x,y) and three reference beacon with coordinates $(x_{i},y_{i})$,  $(x_{j},y_{j})$ and  $(x_{k},y_{k})$, we have that the difference between times of arrival $t_{i}$ and $t_{j}$ to nodes $i$ and $j$, respectively, can be related to the distance between nodes, as expressed in \ref{eq:hyper} \cite{arch_localiz}. $d_{i}$ and $d_{j}$ are the distance from node $i$ and $j$, respectively, to the target object. 

\begin{eqnarray}
& d_{i} - d_{j} = c * (t_{i} - t_{j}) = \sqrt{(x - x{i})^2 + y - y{i})^2} - \sqrt{(x - x{k})^2 + y - y{k})^2}
\label{eq:hyper}
\end{eqnarray}

%________________________________________________________

\section{Positioning Systems}

Positioning systems are used to track the underwater position of a vehicle or other object, in relation to reference structures of transponders called \textit{baseline stations}. These systems are classified based on the distance between the baseline stations. The configurations that will be explained are Long Baseline (LBL), Short Baseline (SBL), Ultra Short Baseline (USBL) and the inverted versions of all above.

\begin{figure}[!htbp]
	\centering
	\includegraphics[width=1\textwidth]{figures/lblsblusbl}
	\caption{Generic configuration of: a) LBL; b) SBL; c) USBL}
	\label{fig:lblsblusbl}
\end{figure}

\subsection{Long Baseline (LBL)}

Long Baseline systems use a positioning method with large distances between baseline stations, with range typically from 50m to more than 2000m  and usually similar to the distance between object and transponders [\hyperref[r:survey-tech-chall]{3}]. A typical LBL configuration is represented in figure a) \ref{fig:lblsblusbl}.

The LBL method uses at least three transponder stations deployed usually on the sea floor, allowing to execute trilateration. Additionally, a transducer is integrated on the object to be tracked. 

A complete localization procedure starts with the vehicle sending an acoustic signal which is received by the transponders. Thereafter the transponders transmit a response and, by analyzing the Time of Flight of the communication, the system can determine the distance between the vehicle and each base station. Then the relative position of the vehicle is determined through trilateration. Additionally, if the transponders have known geographic positions, it is possible to infer the vehicle geographic position. 

As this technique presents large distances between the object and the base stations, the typical 1m to few centimeters accuracy is considered to be high because it will not compromise the localization of the vehicle. 


\subsection{Short Baseline (SBL)}

Short Baseline systems are characterized by having distances around 20m to 50m between baseline stations \cite{survey-tech-chall} and use an operation procedure similar to the LBL method. However, the transponders are usually placed in a moving platform, which assures a fixed relative position between them. A typical SBL configuration is represented in figure b) \ref{fig:lblsblusbl}.

The position of the vehicle to be tracked can be determined by translating the Time of Flight between the multiple transponders and the object into a distance value, which is achieved by equation \ref{eq:sbl} \cite{sbl}. The $t_{i}$ corresponds to the propagation time of the signal from the vehicle to the \textit{i}th transponder, \textit{c} is the speed of sound, [$x_{b_{i}}$, $y_{b_{i}}$ and $z_{b_{i}}$] is the coordinate position of the transponder.

\begin{eqnarray}
&\sqrt{ (x_{b_{i}}-x)^2 + (y_{b_{i}}-y)^2 + (z_{b_{i}}-z)^2 } = c\ *\ t_{i}
\label{eq:sbl}
\end{eqnarray}

In a SBL system, when the distance between baseline stations is increased the accuracy improves and, contrarily, when the mentioned distance decreases the accuracy deteriorates, which can raise some deployment challenges.

\subsection{Ultra Short Baseline (USBL)}

Ultra short baseline systems are composed essentially by one baseline station, with an array consisting of several traducers distanced typically less than the wavelength \cite{lblsblusbl}, and a transponder integrated on the object to be tracked. It is usually used in underwater positioning in shallow areas of the sea, as represented in figure c) \ref{fig:lblsblusbl}.

Similarly to the previously mentioned procedures, the USBL positioning method relies on the Time of Flight of the exchanged signals. However, the traducers are too spatially close from each other to execute an accurate trilateration. Instead, it is measured the phase difference or time-delay difference of the received signal between every traducer, in order to estimate the azimuth and distance to the acoustic source. 

Assuming a three dimensional scenario for the positioning system, as represented in figure \ref{fig:usblgeo}, the object's coordinates are given by equations \ref{eq:usblgeo1}, \ref{eq:usblgeo2} and \ref{eq:usblgeo3} \cite{usbl-new}. The $\lambda$ corresponds to the wavelength of the of the transmitted signal which depends on its operation frequency, \textit{f}, and it is affected by the speed of sound \textit{c}, as represented equation \ref{eq:cfw}.
The \textit{d} represents the distance between hydrophones, $\psi_{12}$ and $\psi_{22}$ are the phase difference between H2 and the other two hydrophones, \textit{H} is the height of the target object, \textit{X} is the distance of the target along the x-axis direction, \textit{Y} is the distance of the target along the y-axis direction and \textit{l} is the slant distance of the target to the hydrophone.
\begin{eqnarray}
& c = f * \lambda
\label{eq:cfw}
\end{eqnarray}
\begin{eqnarray}
& l^2 = X^2 + Y^2 + H^2 
\label{eq:usblgeo1}\\
& \psi_{12} = \frac{2\pi}{\lambda}[\sqrt{l^2} - \sqrt{(d-X)^2 + d^2 + H^2}]
\label{eq:usblgeo2}\\
& \psi_{22} = \frac{2\pi}{\lambda}[\sqrt{l^2} - \sqrt{X^2 + (d-Y)^2 + H^2}]
\label{eq:usblgeo3}
\end{eqnarray}

\begin{figure}[!htbp]
	\centering
	\includegraphics[width=0.5\textwidth]{figures/usbl-config}
	\caption{USBL system configuration}
	\label{fig:usblgeo}
\end{figure}

This is a broadly used technique due to its convenient set up, which allows to have predefined measurements in the order of tens of centimeters and does not require AUV navigation area for the deployment. However it presents the lowest accuracy, comparatively with LBL and SBL, since an error of few centimeters can be realistically corresponding to an inaccuracy of several meters in the position of the object to be tracked.

\subsection{Inverted Systems}

All the previously mentioned positioning techniques use a configuration in which the vehicle to be tracked has a single transducer and there is an external set of transponder to determine the positioning of the said object. However, there is the possibility to benefit from the inverse configuration in some applications. Therefore, there are also the iLBL, iSBL and iUSBL methods, which have the same operation principals as LBL, SBL and USBL, respectively.

%________________________________________________________

\section{Commercial Solutions}

There are several commercial solutions for underwater positioning using the ultra-short baseline method. In this section, it will be presented some of the available devices in the market, indicating their main properties and capabilities. Table \ref{tab:solutions} summarizes the systems with most relevance to the present work. The Medium Frequency (MF) bandwidth is attributed to devices whose manufacturer did not specified the actual frequency range.

\textit{Evologics} produces the S2C R USBL series of acoustic modems \cite{evologics1}, with Sweep Spread Carrier (S2C) technology \cite{evologics2} which uses a broad frequency range to propagate over large distances with reduced noise. The devices have a fixed 0.01m slant range accuracy and a 0.1 degree bearing resolution. These are essentially divided into two groups:
\begin{itemize}
	\item High speed mid-range devices: contains the 18/34 transceivers family \cite{evologics3}, which presents various options for the USBL antenna beam pattern and it is optimal for transmission in horizontal channels.
	\item Depth rated long-range devices: includes the 12/24 transceiver \cite{evologics4}, which have a directional (70 degrees) USBL antenna  and it is optimal for transmission in vertical channels.
\end{itemize}

\textit{Sonardyne} markets the Ranger 2 systems. The Micro-Ranger 2 \cite{sonardyne1} is very easy to use without previous experience and it is appropriate for shallow waters, achieving accuracy of 0.2\%. The Mini-Ranger 2 is ideal for nearshore missions and it is  used for simultaneous tracking of various mobile targets, whose position is updated every 3 seconds.

\textit{Applied Acoustics} offers the Easytrak USBL Systems, which includes the processing software for estimating the position. The Alpha Portable 2655 consists in a very compact structure that includes an array transducer and is capable of reaching a 10cm slant range resolution and a 2 degree RMS.

\textit{Kongsberg} produces the HiPAP family of transducers \cite{hipap_hardw}, which can use the Cymbal acoustic protocol (PSK) or the frequency shift (FSK) modulation technique. Particularly the HiPAP 352 is the model with higher number of active transducers and is able to reaches 0.02m of range accuracy.

\begin{table}[t]
	\centering
	\begin{tabular}{|c|c|c c c|}
		\hline
		Company
		& System
		& Bandwidth(kHz)
		& Connection(kbps)
		& Range(m) \\ \hline 
		\multirow{2}{4em}{Evologics} 
		& S2C R 18/34D USBL & 18-34 & up to 13.9 & 3500\\
		& S2C R 12/24 USBL & 12-24 & up to 9.2 & 6000\\
		\hline 
		\multirow{2}{4em}{Sonardyne} 
		& Micro-Ranger 2 & MF & 0.2-9 & 995\\
		& Mini-Ranger 2 & MF & 0.2-9 & 995\\ 
		\hline 
		\multirow{2}{4em}{Applied Acoustics} 
		& Easytrak Alpha Portable 2655 & MF & n.d. & 500 \\
		&  &  &  & \\
		\hline 
		\multirow{1}{4em}{Kongsberg} 
		& HiPAP 352 & 21-31 & n.d. & 5000 \\
		\hline 
	\end{tabular}
	\caption{Overview of commercial solutions}
	\label{tab:solutions}
\end{table}


%_________________________________________________________


%\begin{eqnarray}
%&10\ log(a(f)) = 0.11(\frac{f^2}{1+f^2})+44(\frac{f^2}{4100+f^2})+2.75*10^{-4}f^2+0.003 
%\label{eq:abs_af}
%\end{eqnarray}


%\begin{eqnarray}
%CIF_1: \hspace*{5mm}F_0^j(a) &=& \frac{1}{2\pi \iota} \oint_{\gamma} \frac{F_0^j(z)}{z - a} dz\\
%CIF_2: \hspace*{5mm}F_1^j(a) &=& \frac{1}{2\pi \iota} \oint_{\gamma} \frac{F_0^j(x)}{x - a} dx 
%\label{eq:cif}
%\end{eqnarray}


\chapter{State of the Art} \label{chap:}


\chapter{Research Problem} \label{chap:problem}

This chapter intends to clarify the problem addressed by the present dissertation. Section \ref{sec:prob-state} presents the details behind the research work as well as the problems it intends to solve. Having this clear, the dissertation hypothesis is stated in section \ref{sec:hypoth-rq} along with the research questions that are the main issues that are being explained with the present document. Lastly, the used validation methods are specified in section \ref{sec:validation}.

%------------------------------------------------------------------------------------
\section{Problem Statement} \label{sec:prob-state}

The Ultra-Short Baseline system is among the most deployed positioning methods using underwater acoustics. There is a vast knowledge of its function and capabilities, therefore its implementation does not constitute a technological innovation nowadays. 
In the considered scenario, previously mentioned in chapter \ref{chap:intro}, an AUV is taking part on a long-term underwater mission in which it periodically sends known signals to the surface with a pinger. In such case, the mule AUV needs to be provided with an USBL system to receive the signal and estimate the position of the other AUV to navigate near it. The simplified communication system is illustrated in figure \ref{fig:auv_scene}. 

\note{GROW illustration}

\begin{figure}[!htbp]
	\centering
	\includegraphics[width=0.8\textwidth]{figures/proposed-solution}
	\caption{Communication System}
	\label{fig:auv_scene}
\end{figure}

This partial USBL system was developed in previous dissertations and research work, which can be better understood in \cite{afonso-thesis}. Briefly, the system consists on a transducer of four hydrophones forming a 3D array deployed on the mule AUV. The distance between AUVs is given by the cross-correlation between the received and expected signals. Since the operation frequency range is around tens of kilohertz, which minimizes the attenuation of sound, the time difference of arrival has to be refined by analyzing the relative phase differences between hydrophones.

Considering that the survey AUV navigates freely trough unknown locations, the USBL system to be employed needs to fulfill particular requirements that common commercial solutions do not comply. 

Firstly, the system needs to be able to cover both short and long range distances, going from tens of centimeters to several hundreds of meters between the receiver and the transmitter, with the best estimation precision possible. For long range positions, the accuracy of the estimation affects how direct is the path for the mule AUV to reach the acoustic source. This influences the overall energy consumption, duration of navigation search and can affect the reliability of the process. For short range position, an accurate estimate allows to avoid collisions and correctly establish chosen relative positions between vehicles. Additionally, an increase in the frequency of position estimation would consume more power but provide more robust positions, which is desirable for short range scenarios. The available market systems usually offer multiple solutions with different limited operation ranges, which would force to employ more than one system to achieve the mentioned range requirement.

Secondly, the USBL system needs to be capable of detecting incoming signals from any position in space, since the localization detection is solely based on the received signals. Since the system is composed by various sensors, it is expected that they are arranged in varied positions and with different perspectives so they can cover a wider area. However, considering they are supposed to be employed on an AUV, the vehicle's body represents an opaque obstacle to signals. Therefore, with only four fixed hydrophones, it is not possible to detect positions with full line of sight, as intended, since not all hydrophones would have line of sight to the transmitter at all times.

The system that is proposed in this dissertation intended to resolve this technological gap with a system that satisfies the described requirements. Since only four hydrophones are enough to obtain a position estimation (bearing and distance), if they assume a fixed position on an AUV's body it would inevitably limit the direction to where the sensor has direct line of sight. Therefore, the suggested method implies deploying multiple sensors in the vehicle. From the available sensors, only four would be used simultaneously to receive the signals and feed them to the processing system. By adopting this concept, a main issue that arises is where to place the hydrophones within the vehicle. This constitutes the main research topic conducted in the present thesis.

Considering that the mentioned mechanism is meant to be applied in mobile vehicles with changing environment conditions, it is useful to integrate it in a system which is responsive in real time. Accordingly, the process that selects four hydrophones among the available set can be transformed into a dynamically reconfigurable system which enables the hydrophones commutation according to the sensors' configuration that minimizes the estimation error.

The study conducted in the scope of this dissertation intends to prove the functionality of the developed method, validate the hypothesis declared in \ref{sec:hypoth-rq} and draw conclusions on the research questions.

%------------------------------------------------------------------------------------
\section{Assumptions} \label{sec:premises}

This research work relies on a set of premises that were considered throughout the development of the proposed system, as presented in this section.

\paragraph{Number of sensors} For the estimation of the position in 3D space, a multilateration approach was used, as explained in \ref{subsubsec:lateration}. Therefore, a minimum of 4 hydrophones are needed so that it is possible to define the position of the transmitter. Using only two sensors, two possibility spheres are formed around these sensors whose intersection originates a circle that contains the location possible solutions. By adding a third sensor, this circle is intersected by another sphere which originates only two location possibilities. Finally, a fourth sensor is added to exactly differentiate which is the accurate location solution.

\paragraph{Synchronism} The system integrates a synchronization mechanism that allows to know the time of emission of an acoustic signal and  Hence it is possible to compute the ToA of the signal which indicates the range between the transmitter and the receiver.

\paragraph{Noise characteristics} The system assumes an injected error $e_i$ added to the time differences of arrival, $ \Delta t_{ij}$. These errors are mutually independent and follow a Gaussian distribution with zero mean and a configurable variance of $\sigma^{2}$, i.e., $e_i \sim \mathcal{N}(0,\,\sigma^{2})$. 

For the simulations performed in this project, a deviation of 5$^{\circ}$, or a window of $[-2.5^{\circ},2.5^{\circ}]$, in phase difference estimation of incoming signals was considered to be reasonable for an underwater navigation scenario. Therefore, since the specified period of the signals is $T = \frac{1}{24400}$, then the 5$^{\circ}$ will be equivalent to $\frac{5^{\circ}}{360^{\circ}}*T$ which is approximately a deviation of $0.5\mu s$. Hence the considered standard deviation $\sigma$ of the error $e_i$ in the computed time differences of arrival is equal to $0.5\mu s$.

\paragraph{Numeric quantization} The digital signal processing system uses a finite quantization resolution with 7 decimal places for the phase difference. Since the added noise expressed in degrees is an integer, which is $10^7$ times larger than the system resolution, then the numeric quantization is not significant in the present work.

\paragraph{Reference axis} The origin of the reference axis is defined at the center of the structure where the hydrophones are fixed, which in this case is the AUV.

\paragraph{Propagation speed} The considered speed of sound is 1500 m/s, which corresponds
to the underwater propagation velocity of waves in typical conditions.

%------------------------------------------------------------------------------------
\section{Hypothesis and Research Questions} \label{sec:hypoth-rq}

This dissertation intends to complement previous research work and answer to a core research hypothesis which serves as fundamental investigation purpose. This research hypothesis can be stated as:
\\

\textit{"Using a USBL system that reconfigures the hydrophone selection leads to an improvement on the underwater localization precision, allowing to always have a set of four active hydrophones with line of sight to the transmitter and makes it suitable for both short and long range estimation."}
\\

Attending the proposed hypothesis, the topics that are intended to be explored and discussed in this thesis's work can be summarized in the following research questions:

%Research Questions
\begin{itemize}

	\item \textbf{RQ1: }\textit{What method should be adopted in order to efficiently compare the performance of hydrophone configurations?}
	
	\item \textbf{RQ2: }\textit{What decision metric(s) should be used to evaluate the optimal hydrophone configuration for a specific angle of arrival?}
	
	\item \textbf{RQ3: }\textit{How should the system be developed in order to assure that the selected hydrophones always have line of sight to the transmitter?}
	
	\item \textbf{RQ4: }\textit{Are there distinct best hydrophone configurations for short and long range estimation?}
	
\end{itemize}

These questions summarize the main topic points which are explored in the scope of this thesis and are the essential inquiries that it intends to answer.

%------------------------------------------------------------------------------------
\section{Validation Methods} \label{sec:validation}

The validation of scientific work is a key factor to demonstrate how reliable and effective it is. In this thesis, three essential methods are used to validate the functionality of the developed techniques:

\begin{itemize}
	
	\item \textbf{Simulation}
	
	The considered immediate approach to evaluate the functionality and behavior of the system consists in creating a set of simulation procedures which are as close as possible to the real environment and the physical system. These simulations were made as MATLAB scripts carefully designed to integrate realistic parameters, such as expected environment noise and other limitations.
	
	\item \textbf{Scientifically recognized methods}
	
	When composing a system, it can be useful comparing the studied approach with widely used methods which are recognized in the scientific community. By doing this, we can gain a level of confidence in the developed system and in the obtained results.
	
	\item \textbf{Field experiments}
	
	After having the analytical methods and simulations coherent, it is essential then to test the system in a real environment in order to assess the functionality and performances when real conditions are added. By testing it in a real application it is possible to take conclusions about its robustness and consider improvements or refinements for the system.
	Due to the exceptional pandemic situation, in the present work it was only possible to perform field tests on the developed digital signal processing module. 

\end{itemize}

\chapter{Ultra-Short Baseline System} \label{chap:proposed_sys}

This chapter is dedicated to the presentation and overall explanation of the developed system, highlighting its capabilities, the used methodologies and overall design strategies. 
The system will be presented in two distinct sections. The first component is the HDL module, which falls into the spectrum of hardware design and requires insight on hardware development and good practices. The second section relies on software development to complement the functionality of the mentioned module, so that is possible to deliver the desired result.

\section{HDL Module Architecture} \label{subchap:HDL module}

The system which is proposed to be implemented in this research work has as input 4 signals which are received by each hydrophone of the array, and outputs an average phase difference between all combinations of pairs of hydrophones. 

- ver regras basicas de hardware development
- sustema sincrono, available clock cycles globais
- hardware limitations
- tamanho das entradas


\begin{enumerate}
	\item Hilbert Filter
	\item Cordic
	\item phasediff
	\item phasemean
\end{enumerate}

\subsection{Hilbert Filter}


\begin{eqnarray}
H(f)(t) = \frac{1}{\pi}\int_{-\infty}^{\infty}\frac{f(\tau)}{t-\tau}d\tau
\label{eq:hilbert_integral}
\end{eqnarray}

\begin{eqnarray}
&Imag_0 = x_{-1}*c_1 + x_{-3}*c_3 + x_{-5}*c_5 + x_{-7}*c_7
\label{eq:hilbert_imeq}
\end{eqnarray}

\begin{eqnarray}
&Real_0 = x_{-4} 
\label{eq:hilbert_reeq}
\end{eqnarray}

- matematica brevemente, equação base, resposta impulsional, ganho, coeficientes e ordem usada
\\
- schematics 
\\
- explicar design decisions
\\
- descrever brevemente flow do sinal no hardware

\subsection{Cordic}
- descriçao do que faz, matematica (?)
\\
- entradas e saídas, clocks, ROM

\subsection{phasediff}
-pequeno esquema 
\\
- 1 sub

\subsection{phasemean}
- pequeno esquema
\\
- N accumulated
\\
---------
\\
apresentar esquema global menos pormenorizado


\subsection{Phase Ambiguity}

When the information about the time of flight of a signal is available, it is relatively easy to estimate the range of the communication since there can be a direct conversion between them. However, when dealing with phase differences, there is no exact time notion, so it is necessary to start by defining a reference point. 

Considering sinusoidal signals, when we have an array with four hydrophones spatially placed to form a 3D layout, the signal that is arriving to each  hydrophone in different times consequently have different phases. However, since sinusoidal signals are periodic, this means that for different signal periods the same phase value is observed, i.e. the phase is ambiguous. It is possible to observe this phenomenon in figure \ref{fig:phasediff}. In this illustration, $\alpha$ represents the observable phase difference of hydrophone $H_4$ to the reference point $H_1$. However, the actual phase difference which is intended to obtain, $\Delta p_4$, is one period of the signal, $\lambda$, added to the observable phase $\alpha$.

\begin{figure}[!htbp]
	\centering
	\includegraphics[width=1.2\textwidth]{figures/phase-diff}
	\caption{Phase difference to reference point and phase ambiguity}
	\label{fig:phasediff}
\end{figure}

For this reason, it is crucial to consider that the phase difference is given by the obtained phase value added by the number of periods ahead from the considered reference period.

In the system under study, the sent signals work with a operation frequency of 24.4 $kHz$. The corresponding signal period is $T = \frac{1}{24400} $ seconds which, considering the underwater acoustic speed \textit{c} equal to a standard 1500 $m/s$, the wavelength is approximately equal to $\lambda = \frac{T}{c} = 6.1 cm$. Having this into consideration, after obtaining the time of arrival to each hydrophone given by the cross correlation instances, besides the reference one, it is possible to conclude if the phase shift is superior to one period by analyzing if the time difference is greater than the duration of one period \textit{T}. In figure \ref{fig:phasediff}, each mentioned time difference between $H_1$ and $H_2$, $H_3$ and $H_4$ is converted to the corresponding phase differences $\Delta p_2$, $\Delta p_3$ and $\Delta p_4$.

%From this phase differences, it is possible to estimate the angle of arrival from which the acoustic wave is coming from by comparing all pairs of hydrophones: H1-H2, H1-H3, H1-H4, H2-H3, H2-H4, H3-H4. 

One possibility to solve phase ambiguity in this system would be to place the four hydrophones with a baseline spacing inferior to $\frac{1}{2}$ of a wavelength, since the maximum reached by phase difference is 180 degrees. This way it would be possible to immediately deduce the phase difference since it would always be contained in one period. However, positioning the hydrophones closer together leads to smaller  values,causing a consequent increase on the estimation error due to varying environment conditions (briefly enumerated in \ref{subsec: acoustic-channel}). Additionally, since the hydrophones to be used in this system have a corresponding diameter of roughly half of a wavelength, they would not allow to execute the mentioned configuration and so this possibility will not be contemplated.

In order to compensate this phase ambiguity, a simple relation was developed which allows to calculate the absolute time difference between the moment a signal is received by hydrophone A, $T_A$, and when the same signal is received by a further hydrophone B, $T_B$. Figure \ref{fig:ambiguity} illustrates this association, where the represented sinusoidal waves correspond to the same signal arriving at hydrophones A and B. This correspondence uses the time stamps obtained by the correlation peaks combined with the calculated phase difference, that is determined in parallel, so that the measurement is more accurate. Equation \ref{eq:phase-amb} translates this relation, where $t_1$ and $t_2$ are the correlation peaks obtained from the signal arriving at hydrophone A and B, respectively, and so by rounding for the next integer number the difference between the correlation peaks, $t_2 - t_1$, we will obtain in which period, $T$, of signal in A will the signal in B arrive. Then the measurement is improved by subtracting a phase difference, $\theta_B - \theta_A$, so that the instant in which the signal is detected in hydrophone B can be defined. 

\begin{figure}[!htbp]
	\centering
	\includegraphics[width=0.6\textwidth]{figures/ambiguity}
	\captionsetup{justification=centering,margin=2cm}
	\caption{Ambiguity correction through correlation and phase difference}
	\label{fig:ambiguity}
\end{figure}

\begin{eqnarray}
& T_B - T_A = round(\frac{t_2-t_1}{T}) - (\theta_B - \theta_A)
\label{eq:phase-amb}
\end{eqnarray}


\section{Position Estimator} \label{subsec:AoA}

The proposed position estimator uses vector algebra, the phase differences obtained from the system described in \ref{subchap:HDL module}, synchronization elements and additional mechanisms that will be further explained in the present section.

\subsection{Preliminary considerations}

\subsubsection{Number of sensors}
For the estimation of the position in 3D space, a multilateration approach was used. As explained in \ref{subsec:multilateration}, the concept of multilateration combines the information of the relative distances between multiple sensors and a target in order to locate it. 

In the present case, a total of four sensors are needed so that it is possible to define the position of target. Using only two sensors, two possibility spheres are formed around these sensors whose intersection originates a circle that contains the location possible solutions. By adding a third sensor, this circle is intersected by another sphere which originates only two location possibilities. Finally, a fourth sensor is added so that it is possible to exactly differentiate which one of the two final solutions is the accurate location solution. 

\subsubsection{Position in relation to ToA}
%---Hyperbole------------------------------
To better understand the location estimation of an acoustic source in relation to the position of a pair of hydrophones, we can initially adopt the two dimensional scenario of figure \ref{fig:hyper}. 

Considering two hydrophones at known relative positions $(-f,0)$ and $(f,0)$, we can model all possible acoustic source locations for a specific ToA through hyperbolas. This is due to the fact that, by definition, the sum of the distances from the focus of each hyperbole, where each hydrophone is placed, to any point of the hyperbolic geometry corresponds to a constant value. 
This means that, in figure \ref{fig:hyper}, any point $(x,y)$ that is contained in the hyperbole corresponds to a constant $|d_2-d_1|$ value which, after some formulation, is in fact equal to $2*v$ or the distance between the vertexes of each hydrophone's hyperbola. Therefore, it is possible to trace a hyperbole that represents the positions of the target in space both based on their distance and the signal's ToA. In the exceptional case where $d_3=d_4$, we can observe that the possible positions are represented by an equidistant straight line to each hydrophone, such as the y axis.

\begin{figure}[!htbp]
	\centering
	\includegraphics[width=0.8\textwidth]{figures/hyperbole-dist}
	\captionsetup{justification=centering,margin=2cm}
	\caption{Hyperbolic representation of acoustic source position possibilities in relation to ToA to two hydrophones}
	\label{fig:hyper}
\end{figure}

\subsubsection{ToA approximation}

In order to estimate the location of an acoustic source we take into account the phase differences between each pair of hydrophones, carefully explained in section \ref{subchap:HDL module}. These phase differences can be translated into periods of the signal which combined with the ToA obtained from correlation of arriving acoustic signals are equivalent to relative distances.

Following the previous idea, it is possible to model the distance of one sensor to the target based on the known distance of a second sensor to the same target. 
This is to say that for two sensors with known relative positions where hydrophone 1 is the closer to the target, the distance from hydrophone i to the target, $D_i$, can be expressed as the distance of hydrophone j to the target, $D_j$, added by the time difference of arrival, $\Delta t_{ij}$, multiplied by the propagation velocity, $c_s$. Overall, this relationship is declared in equation \ref{eq:dist_to_target}.

\begin{eqnarray}
& D_i = D_j + \Delta t_{ij} * c_s
\label{eq:dist_to_target}
\end{eqnarray}

Therefore, the same logic can be applied for multiple hydrophones. In the present work, in which it is considered a system with four hydrophones, a synchronization mechanism allows to determine the signals' ToA between the transceiver and the hydrophones. However, in order to simplify the synchronicity and decrease errors that arise from it, the module that precisely computes the phase differences of the received signal in the hydrophones is used so that is possible to apply the relationship in \ref{eq:dist_to_target}. Consequently, a better angle of arrival estimation can be achieved when using this approximation than if all four times of flight are used for the same purpose.

\subsection{Methodological Approach}

The goal of the proposed system is to estimate the position of an acoustic source in relation to known positions of a configuration of sensors, in a system of geometric axes with a defined origin. For this purpose, the logic employed is based on vector algebra with other physical considerations, detailed in the present subsection. 

Figure \ref{fig:AoA-init} represents the schematic of a considered scenario, where four hydrophones are placed in known relative positions in space and the origin of the axis is set on the body of the AUV or an alternative fixed structure. Then $r_i$ is defined as the vector that connects the origin of the axis to hydrophone $i$ and $rr_i$ defines the vector that connects hydrophone $i$ to the acoustic source. The black cross represents the acoustic source which is located somewhere in space. At last, the subtraction of the mentioned vectors is equal to $r$, according to \ref{eq:sum-vec}, which corresponds to the position of the acoustic source in relation to the origin of the axis and, overall, it is the variable that the method aims to determine.

\begin{eqnarray}
& r_i = r + rr_i
\label{eq:sum-vec}
\end{eqnarray}

\begin{figure}[!htbp]
	\centering
	\includegraphics[width=0.8\textwidth]{figures/AoA-init}
	\captionsetup{justification=centering,margin=2cm}
	\caption{Considered scheme for angle of arrival estimation}
	\label{fig:AoA-init}
\end{figure}

Then we can define the times of arrival to each hydrophone as \ref{eq:toa-4h}, where $t_0$ is the absolute time of emission, $c_s$ is the underwater sound speed and $\rho_i$ is the norm of $rr_i$, \ref{eq:rho}, which translates to the distance from hydrophone $i$ to the acoustic source. 

\begin{eqnarray}
& t_i = t_0 +  \frac{\rho_i}{c_s}
\label{eq:toa-4h}
\end{eqnarray}

 However, as explained in the previous section, instead of using the absolute ToA in each hydrophone by computing the expression \ref{eq:toa-4h} for each of them, it can be expressed as a function of a single reference ToA. A simple logic was applied in order to determine this reference hydrophone, which starts by identifying the closest to the acoustic source. This allows to obtain all relative times of arrival by adding the defined reference time to each  between a hydrophone and the reference one. This is achieved by analyzing the  of each pair, $ \Delta t_{ij}$, for all possible combinations of two among four hydrophones, making up a total of six combinations. Considering each hydrophone pair $ij$ with $i, j= \{1,2,3,4\}$ :
 
 \begin{itemize}
 	\item if $ \Delta t_{ij}$ is positive, then hydrophone $i$ is closer to the acoustic source
 	\item if $ \Delta t_{ij}$ is negative, then hydrophone $j$ is closer to the acoustic source
 	\item if $ \Delta t_{ij}$ is zero, then $i$ and $j$ hydrophones are equidistant to the acoustic source
 \end{itemize}
 
Considering these relations, it is possible to compose a vector that accumulates the closer hydrophone between each pair for a certain position of the acoustic source. Extracting the mode of this vector will then return the chosen hydrophone in most cases and therefore the overall closer to the acoustic source. If the closer hydrophones are the equidistant to the source, then it is indifferent which one is selected.  

Thereafter, recalling expression \ref{eq:dist_to_target}, it is possible to write \ref{eq:toa_relation2}, \ref{eq:toa_relation3} and \ref{eq:toa_relation4} which translate the used relations, where  the chose reference sensor is hydrophone 1, for the purpose of exemplification.

\begin{eqnarray}
& T_2 = T_1 + \Delta t_{12} * c_s
\label{eq:toa_relation2}\\
& T_3 = T_1 + \Delta t_{13} * c_s
\label{eq:toa_relation3}\\
& T_4 = T_1 + \Delta t_{14} * c_s
\label{eq:toa_relation4}
\end{eqnarray}

If then the distance $\rho_i$ is raised to the power of two, we know that $||rr_i||^2 = r_i^{T}r_i$, which allows to deduce equation \ref{eq:rho1} after some mathematical manipulation. Considering $\rho_i$ a physical distance, it is also possible to express it trough equation \ref{eq:rho2}, which uses the speed of propagation underwater multiplied by the ToA of the signal from the acoustic source to hydrophone $i$.

\begin{eqnarray}
& \rho_i = ||rr_i|| 
\label{eq:rho}\\
&\rho_i^{2} =  r^{T}r + 2r^{T}r_i + r_i^{T}r_i
\label{eq:rho1}\\
&\rho_i^{2} = c_s^{2} (t_i-t_0)^{2}
\label{eq:rho2}
\end{eqnarray}

Since two distinct relations are defined for $\rho_i^{2}$, then it is possible to consider the algebraic expressions as equivalent, thus forming a single equation to be resolved with only one unknown variable. After some mathematical manipulation, the matrix relation \ref{eq:AoA-matrix} is achieved, where $r$ is isolated and can be estimated.

\begin{eqnarray}
\begin{bmatrix}
1 & 2\: r_i^{T}
\end{bmatrix}
\begin{bmatrix}
r^{T} r \\
r
\end{bmatrix}
=  
\begin{bmatrix}
c_s^{2} (t_i-t_0) - r_i^{T} r_i
\end{bmatrix}
\label{eq:AoA-matrix}
\end{eqnarray}
 
In order to resolve this system of equations and isolate $r$, the least squares method is applied. If \ref{eq:AoA-matrix} is extended to the four considered hydrophones, we obtain matrix $A$ represented as \ref{eq:A} and $Y$ equivalent to \ref{eq:Y}. Then the least squares method is expressed as  \ref{eq:least-square}, where $X$, $\mathbb{R}^{4}$, holds the Cartesian result of $r$. As the method formulates four equations that are meant to calculate only three coordinates, $X$ will contain a fourth element that consists on a nonlinear component to $||r||^{2}$. This causes the estimator to be considered not efficient.

\begin{eqnarray}
& A = 
\begin{bmatrix}
1 & 2\: r_1^{T}\\
1 & 2\: r_2^{T}\\
1 & 2\: r_3^{T}\\
1 & 2\: r_4^{T}
\end{bmatrix}
\label{eq:A}
\end{eqnarray}

\begin{eqnarray}
& Y = 
\begin{bmatrix}
c_s^{2}\: (t_1-t_0)^2 - r_1^{T} r_1\\
c_s^{2}\: (t_2-t_0)^2 - r_2^{T} r_2\\
c_s^{2}\: (t_3-t_0)^2 - r_3^{T} r_3\\
c_s^{2}\: (t_4-t_0)^2 - r_4^{T} r_4\\
\end{bmatrix}
\label{eq:Y}
\end{eqnarray}

\begin{eqnarray}
& X = (A^{T}*A)^{-1}*A^{T}*Y
\label{eq:least-square}
\end{eqnarray}

It is important to notice that the $A$ matrix has to be invertible, thus the rows which contain the chosen hydrophone configuration have to be linearly independent.

After infer the Euclidean vector $r$, it is possible to obtain both the bearing trough its direction, $\hat{\boldsymbol{r}}$, and the range through its magnitude, $||r||$.

\subsection{Precision analysis}  \label{subchap:precision-analy}

A methodology was formulated in order to evaluate the precision that the estimator can achieve in defined circumstances. For this initial approach to the study, the following conditions are considered: 

\begin{enumerate}[label=\alph*)]
	\item \textbf{Sensor Configuration}  
	
	Each hydrophone configuration is analyzed individually. It is a parameter to be always defined and known from the begging of each simulation.
	
	\item \textbf{Reference axis}
	
	 The origin of the reference axis is defined at the center of mass of the structure where the hydrophones are fixed, which in this case is the AUV.
	
	\item \textbf{Acoustic source position} 
	
	The considered positions for the acoustic source are defined in spherical coordinates. Thus the source's range is selected in accordance with the norm, $n$, whereas the azimuth, $\phi$, and elevation, $\theta$, define the angle of arrival of the received signal. Moreover, the azimuth component covers the interval [-180$^{\circ}$, 180$^{\circ}$] in steps of one and the elevation component covers the interval [-90$^{\circ}$ to 90$^{\circ}$] in steps of one, forming spheres with the selected norm around the reference axis' origin.

	\item \textbf{Injected error} 
	
	In order to make the study more realistic, an $e_i$ error is added to the time differences of arrival, $ \Delta t_{ij}$. These errors are mutually independent and follow a Gaussian distribution with zero mean and a configurable variance of $\sigma^{2}$, i.e., $e_i \sim \mathcal{N}(0,\,\sigma^{2})$. 
	
	 For the simulations performed in this project, a deviation of 5$^{\circ}$,or a window of $[-2.5^{\circ},2.5^{\circ}]$, in the angle of arrival estimation was considered to be reasonable for an underwater navigation scenario. Therefore, since the specified period of the signal is $T = \frac{1}{24400}$ and one period corresponds to a 360$^{\circ}$ phase shift, then the 5$^{\circ}$ will be equivalent to $\frac{5^{\circ}}{360^{\circ}}*T$ which is approximately a deviation of $0.5\mu s$. Hence the considered standard deviation $\sigma$ of the error $e_i$ in the computed time differences of arrival is equal to $0.5\mu s$.

	\item \textbf{Propagation speed}
	
	In all performed simulations, the considered speed of sound is $1500 \; m/s$, which corresponds to the underwater propagation velocity of waves in typical conditions.
	
\end{enumerate}

Having the conditions enumerated, the logic of the algorithm occurs as follows. For every defined position of the acoustic source, $s$, a function that consists on the estimator is called, receiving as input the $s$, the positions of the hydrophone configuration, $r_i$, and an injected error in the . It then returns the estimated position of the source in Cartesian coordinates, $[x,y,z]$, and in spherical coordinates, $[n, \phi, \theta]$. As the position $s$ in Cartesian corresponds to the real value that is intended to be estimated, we can also obtain the real spherical coordinates by directly converting $s$ using the Cartesian to spherical relations in \ref{eq:cart2sph}.

\begin{eqnarray}
\begin{cases} 
n =  \sqrt{x^2 + y^2 + z^2}\\ 
\phi  = arctan \frac{y}{x}\\ 
\theta =  arctan \frac{\sqrt{x^2+y^2}}{z}
\end{cases}
\label{eq:cart2sph}
\end{eqnarray}

Consequently all conditions are met to analyze the achieved error in each coordinate by comparing the real position to the estimated values as \ref{eq:error1}, where the tested coordinates are $x, y, z, n, \phi$ and $\theta$.

\begin{eqnarray}
&error_{coordinate} = |estimated_{coordinate} - real_{coordinate}|
\label{eq:error1}
\end{eqnarray}

The metrics used to evaluate the quality of the estimator were :

\begin{itemize}
	\item \textbf{Mean squared error} (MSE): Incorporates both the variance and the bias of the estimator and indicates its overall quality
	\item \textbf{Standard deviation of the error} ($\sigma$) : Indicates how disperse are the estimates from the expected value
	\item \textbf{Minimum error ($min(e_i)$)} : Indicates the minimum error that is obtained by the estimator
%	\item \textbf{Maximum error ($max(e_i)$)} : Indicates the maximum error that is obtained by the estimator
\end{itemize} 

\subsection{Simulations and Conclusions}

A series of simulations were performed in order to understand the behavior and capabilities of the estimator and analyze its overall precision.
%such as a vehicle is moving towards an acoustic signal transmitter,

To illustrate a scenario where this estimator is applicable, we can consider that a vehicle is moving towards an acoustic signal transmitter whose position is unknown. Imagining that the target is at a considerable distance, then the main focus is to achieve an optimal bearing estimation which provides a more direct path and saves resources. The range estimation serves as secondary measurement that indicates how near the vehicle is from the destination, so that it is possible to make control decisions such as moderate the navigation speed in the proximity of the target. For the reasons outlined, the study that follows presents a more thorough analysis of the azimuth and elevation errors. 

In these simulations, two different hydrophone configurations are considered , A and B defined in \ref{tab:configs_test1}, where the columns $r_{Ai}$ and $r_{Bi}$ contain the position's coordinates of each hydrophones $i$.

\begin{table}[!htbp] %use H to adjust
	\begin{center}
		\begin{tabular}{ l | c c c c | c c c c }
			%\hline
			%\multicolumn{1}{c|}{} & \multicolumn{4}{c|}{A} & \multicolumn{4}{c|}{B} \\
	    	\toprule
	       % \cline{2-9}
			\multicolumn{1}{c|}{} & $r_{A1}$ & $r_{A2}$ & $r_{A3}$ & $r_{A4}$ & $r_{B1}$ & $r_{B2}$ & $r_{B3}$ & $r_{B4}$ \\
			\midrule
			\multirow{1}{0.5em}{x} 
			& 0.02 & 0.02 & 0 & 0 & 0.1 & 0 & 0 & 0 \\
			%\hline 
			\multirow{1}{0.5em}{y} 
			& 0 & 0 & 0.1 & -0.1 & 0 & 0 & 0.05 & -0.05\\
			%\hline 
			\multirow{1}{0.5em}{z} 
			& 0.1 & -0.1  & 0 & 0 & 0 & 0.1 & -0.1 & -0.1\\
			\bottomrule 
		\end{tabular}
		\caption{Hydrophone configurations for precision tests}
		\label{tab:configs_test1}
	\end{center}
\end{table}

Considering the conditions previously presented, tests were ran in order to understand the error magnitudes that are achieved in different scenarios by the position estimator.

For the first simulation, configuration A is tested according to the acoustic source positions consideration described in the previous subsection.

\begin{figure}[!htbp]
	
	\makebox[\textwidth][c]{\includegraphics[width=1.2\textwidth]{figures/plots/plot-s1-A-n10}}
	\captionsetup{justification=centering,margin=2cm}
	\caption{Plots of azimuth, elevation and norm errors for defined source positions}
	\label{fig:s1-A-n10}
\end{figure}


%However, in reality when elevation is -90$^{\circ}$ or 90$^{\circ}$, the azimuth angle has no significance and should not be considered. For that reason, in these simulations the elevation values are limited to an interval between -80$^{\circ}$ and 80$^{\circ}$ so that the metrics to be analyzed present a result that is not influenced by the errors originated from this phenomenon.

%It should be pointed out that while analyzing these parameters, it is expected that for position's elevations in the surroundings of -90$^{\circ}$ and 90$^{\circ}$, large azimuth errors are returned. This is explained  elevation is maximum, the azimuth component has no significance.

%it is expected that in cases where the position's azimuth is in the surroundings of -180$^{\circ}$ and 180$^{\circ}$, the returned error is large since the subtraction between the estimated and the real values can reach near $360^{\circ}$. This happens due to the deviations in the estimate provoked by the injected error, however a correction to this issue was implemented so that the real difference . It is also


Therefore, tables \ref{tab:azimuth-test1}  and \ref{tab:elevation-test1} illustrate the achieved results of azimuth error for several chosen norms.

\begin{table}[!htbp] %use H to adjust
	\begin{center}
		\begin{tabular}{ c | c c c c c }
			%\hline
			\toprule
			% \cline{2-9}
			\multicolumn{1}{c|}{Configuration} & Norm & MSE & Standard Deviation & Minimum \\
			\midrule
			\multirow{2}{*}{A} &10 & 0.5440 & 0.6050 & 8.7434$\times10^{-7}$\\
		%	&100 & 0.5465 & 0.6090 & 6.0529$\times10^{-6}$\\
			&1000 & 0.5425 & 0.5992 &  8.3781$\times10^{-6}$\\
			\midrule			
			\multirow{2}{*}{B} &10 & 0.2300 & 0.2129 & 1.3049$\times10^{-5}$\\
		%	&100 & 0.2304 & 0.2119  & 4.3874$\times10^{-6}$\\
			&1000 & 0.2230 & 0.2115 & 4.0963$\times10^{-6}$\\
			\bottomrule 
		\end{tabular}
		\caption{Comparison of obtained azimuth errors for different configurations}
		\label{tab:azimuth-test1}
	\end{center}
\end{table}

\begin{table}[!htbp] %use H to adjust
	\begin{center}
		\begin{tabular}{ c | c c c c c }
			%\hline
			\toprule
			% \cline{2-9}
			\multicolumn{1}{c|}{Configuration} & Norm & MSE & Standard Deviation & Minimum \\
			\midrule
			\multirow{2}{*}{A} &10 & 0.5440  & 0.1856 & 3.1025$\times10^{-6}$  & \\
		%	&100 & 0.5465 &  0.1845 &  2.1818$\times10^{-6}$ & \\
			&1000 & 0.5425 & 0.1854 & 5.0595$\times10^{-6}$ & \\
			\midrule			
			\multirow{2}{*}{B} &10 & 0.2300 & 0.0668 & 2.7903$\times10^{-5}$ \\
		%	&100 & 0.2304 & 0.0662 & 1.0184$\times10^{-5}$\\
			&1000 & 0.2230 & 0.0663 & 4.3935$\times10^{-6}$ \\
			\bottomrule 
		\end{tabular}
		\caption{Comparison of obtained elevation errors for different configurations}
		\label{tab:elevation-test1}
	\end{center}
\end{table}


\subsubsection{Influence of quantization on precision}

The first term to be analyzed is how much does the quantization of the calculations influence the obtained precision of the estimator. In order to analyze this, a simple adaptation was made to the numeric precision of the  values that are input of the system. Instead of using the MATLAB precision of fifteen decimal places, the value was truncated to a specified number of decimal places, $\kappa$.
Since the time differences of arrival have magnitudes around microseconds, then initially the time differences of arrival are multiplied by $10^6$ to avoid missing information. Then the relation \ref{eq:trucate} is applied resulting in a truncated value of  with $\kappa$ decimal places.

\begin{eqnarray}
&_{truncated} = \frac{round(*2^{\kappa})}{2^{\kappa}};
\label{eq:trucate}
\end{eqnarray}

Finally after the truncation, the value is converted again to seconds to be used in the algorithm.


\subsubsection{Influence of ToA measurement on position estimation}

 \note{ por simulação, conclui que para distancias muito longe (quantizar) não faz diferença ter o TOA e basta os }

\subsubsection{Influence of displacing a specific hydrophone}


\chapter{Experiments on Localization Reliability and Improvement}  \label{chap:study}

 

%% Comment next 2 commands if numbered appendices are not used
\appendix
\chapter{Complementary Information} \label{ap1:Lorem}


\section{Hydrophone configurations numeration}

In table \ref{tab:long} it is possible to consult the composition of the configurations which are mentioned throughout the document. It includes the configuration number and the respective integrated hydrophones.

\begin{longtable}{c | c c c c}
	\caption{Configurations for the Monte Carlo approach with 9 employed hydrophones} \label{tab:long} \\
	
	%\hline \multicolumn{1}{c|}{Configuration number} & \multicolumn{4}{c}{Hydrophones} \\ \hline 
	\endfirsthead
	
	\multicolumn{5}{c}%
	{{\bfseries \tablename\ \thetable{} }} \\
	\midrule\multicolumn{1}{c|}{Configuration number} & \multicolumn{4}{c}{Hydrophones} \\ \hline 
	\endhead
	
	%\midrule \multicolumn{5}{|r|}{{Continued on next page}} \\ \midrule
	\endfoot
	 
	\endlastfoot

			\toprule
			\multicolumn{1}{c|}{Configuration number} & \multicolumn{4}{c}{Hydrophones}  \\
			\midrule 
			\multicolumn{1}{c|}{1} & 1 & 2 & 3 & 4 \\ 
			\midrule 
			\multicolumn{1}{c|}{2} & 1 & 2 & 3 & 5 \\ 
			\midrule 
			\multicolumn{1}{c|}{3} & 1 & 2 & 3 & 6 \\ 
			\midrule 
			\multicolumn{1}{c|}{4} & 1 & 2 & 3 & 7 \\ 
			\midrule 
			\multicolumn{1}{c|}{5} & 1 & 2 & 3 & 8 \\ 
			\midrule 
			\multicolumn{1}{c|}{6} & 1 & 2 & 3 & 9 \\ 
			\midrule 
			\multicolumn{1}{c|}{7} & 1 & 2 & 4 & 5 \\ 
			\midrule 
			\multicolumn{1}{c|}{8} & 1 & 2 & 4 & 6 \\ 
			\midrule 
			\multicolumn{1}{c|}{9} & 1 & 2 & 4 & 7 \\ 
			\midrule 
			\multicolumn{1}{c|}{10} & 1 & 2 & 4 & 8 \\ 
			\midrule 
			\multicolumn{1}{c|}{11} & 1 & 2 & 4 & 9 \\ 
			\midrule 
			\multicolumn{1}{c|}{12} & 1 & 2 & 5 & 6 \\ 
			\midrule 
			\multicolumn{1}{c|}{13} & 1 & 2 & 5 & 7 \\ 
			\midrule 
			\multicolumn{1}{c|}{14} & 1 & 2 & 5 & 8 \\ 
			\midrule 
			\multicolumn{1}{c|}{15} & 1 & 2 & 5 & 9 \\ 
			\midrule 
			\multicolumn{1}{c|}{16} & 1 & 2 & 6 & 7 \\ 
			\midrule 
			\multicolumn{1}{c|}{17} & 1 & 2 & 6 & 8 \\ 
			\midrule 
			\multicolumn{1}{c|}{18} & 1 & 2 & 6 & 9 \\ 
			\midrule 
			\multicolumn{1}{c|}{19} & 1 & 2 & 7 & 8 \\ 
			\midrule 
			\multicolumn{1}{c|}{20} & 1 & 2 & 7 & 9 \\ 
			\midrule 
			\multicolumn{1}{c|}{21} & 1 & 2 & 8 & 9 \\ 
			\midrule 
			\multicolumn{1}{c|}{22} & 1 & 3 & 4 & 5 \\ 
			\midrule 
			\multicolumn{1}{c|}{23} & 1 & 3 & 4 & 6 \\ 
			\midrule 
			\multicolumn{1}{c|}{24} & 1 & 3 & 4 & 7 \\ 
			\midrule 
			\multicolumn{1}{c|}{25} & 1 & 3 & 4 & 8 \\ 
			\midrule 
			\multicolumn{1}{c|}{26} & 1 & 3 & 4 & 9 \\ 
			\midrule 
			\multicolumn{1}{c|}{27} & 1 & 3 & 5 & 6 \\ 
			\midrule 
			\multicolumn{1}{c|}{28} & 1 & 3 & 5 & 7 \\ 
			\midrule 
			\multicolumn{1}{c|}{29} & 1 & 3 & 5 & 8 \\ 
			\midrule 
			\multicolumn{1}{c|}{30} & 1 & 3 & 5 & 9 \\ 
			\midrule 
			\multicolumn{1}{c|}{31} & 1 & 3 & 6 & 7 \\ 
			\midrule 
			\multicolumn{1}{c|}{32} & 1 & 3 & 6 & 8 \\ 
			\midrule 
			\multicolumn{1}{c|}{33} & 1 & 3 & 6 & 9 \\ 
			\midrule 
			\multicolumn{1}{c|}{34} & 1 & 3 & 7 & 8 \\ 
			\midrule 
			\multicolumn{1}{c|}{35} & 1 & 3 & 7 & 9 \\ 
			\midrule 
			\multicolumn{1}{c|}{36} & 1 & 3 & 8 & 9 \\ 
			\midrule 
			\multicolumn{1}{c|}{37} & 1 & 4 & 5 & 6 \\ 
			\midrule 
			\multicolumn{1}{c|}{38} & 1 & 4 & 5 & 7 \\ 
			\midrule 
			\multicolumn{1}{c|}{39} & 1 & 4 & 5 & 8 \\ 
			\midrule 
			\multicolumn{1}{c|}{40} & 1 & 4 & 5 & 9 \\ 
			\midrule 
			\multicolumn{1}{c|}{41} & 1 & 4 & 6 & 7 \\ 
			\midrule 
			\multicolumn{1}{c|}{42} & 1 & 4 & 6 & 8 \\ 
			\midrule 
			\multicolumn{1}{c|}{43} & 1 & 4 & 6 & 9 \\ 
			\midrule 
			\multicolumn{1}{c|}{44} & 1 & 4 & 7 & 8 \\ 
			\midrule 
			\multicolumn{1}{c|}{45} & 1 & 4 & 7 & 9 \\ 
			\midrule 
			\multicolumn{1}{c|}{46} & 1 & 4 & 8 & 9 \\ 
			\midrule 
			\multicolumn{1}{c|}{47} & 1 & 5 & 6 & 7 \\ 
			\midrule 
			\multicolumn{1}{c|}{48} & 1 & 5 & 6 & 8 \\ 
			\midrule 
			\multicolumn{1}{c|}{49} & 1 & 5 & 6 & 9 \\ 
			\midrule 
			\multicolumn{1}{c|}{50} & 1 & 5 & 7 & 8 \\ 
			\midrule 
			\multicolumn{1}{c|}{51} & 1 & 5 & 7 & 9 \\ 
			\midrule 
			\multicolumn{1}{c|}{52} & 1 & 5 & 8 & 9 \\ 
			\midrule 
			\multicolumn{1}{c|}{53} & 1 & 6 & 7 & 8 \\ 
			\midrule 
			\multicolumn{1}{c|}{54} & 1 & 6 & 7 & 9 \\ 
			\midrule 
			\multicolumn{1}{c|}{55} & 1 & 6 & 8 & 9 \\ 
			\midrule 
			\multicolumn{1}{c|}{56} & 1 & 7 & 8 & 9 \\ 
			\bottomrule 
			
		\end{longtable}


%%----------------------------------------
%% Final materials
%%----------------------------------------

%% Bibliography
%% Comment the next command if BibTeX file not used, 
%% Assumes that bibliography is in ``myrefs.bib''
\PrintBib{myrefs}

%% Index
%% Uncomment next command if index is required, 
%% don't forget to run ``makeindex tese'' command
%\PrintIndex

\end{document}
