\chapter*{Resumo}
%\addcontentsline{toc}{chapter}{Resumo}
Dispositivos robóticos programáveis como \textit{Autonomous Underwater Vehicles} (AUVs) são excelentes meios para exploração subaquática, já que são capazes de executar missões de longa duração com variadas possibilidades de aplicação e objetivos. Neste sentido, o conceito de usar AUVs como "mulas" transportadoras de dados surgiu como forma de antecipar o acesso aos dados recolhidos por AUVs durante missões autónomas de longa duração.

A presente dissertação foca-se no estudo de mecanismos que conduzam à melhoria da flexibilidade e da precisão de um sistema USBL experimental para uso num AUV de pequeno porte e destinado a operar em curto e longo alcance. Primeiro, é descrito o design da arquitetura de um modulo capaz de melhorar a precisão da medida dos tempos de chegada de sinais enviados por uma fonte acústica. De seguida, é conduzido um estudo sobre possíveis métodos de avaliação do desempenho de uma configuração de sensores, já que consiste num fator crucial na precisão de estimação. Por último, o método de seleção adaptativa de configurações é apresentado, o qual serve como ferramenta que reconfigura o conjunto de hidrophones ativos, a partir de um grupo discreto em posições fixas conhecidas, dependendo da localização estimada do transmissor. Este método pretende alcançar uma maior precisão na localização e retificar problemas que surgem em sistemas USBL clássicos. 

Após a implementação, todos os mecanismos desenvolvidos foram sujeitos a testes detalhados em simulação que validam o seu funcionamento e demonstram resultados promissores em condições controladas. Adicionalmente, foram realizados ensaios preliminares em ambiente laboratorial mas os testes de campo ficaram muito aquém do desejável devido às limitações causadas pelo estado atual de pandemia.

\chapter*{Abstract}
%\addcontentsline{toc}{chapter}{Abstract}
Robotic programmable devices such as Autonomous Underwater Vehicles (AUVs) are great means for underwater exploration, as they are capable of executing long term missions with many possible applications and goals. In this regard, the concept of using AUVs as "mules" for data transport appeared as a way to anticipate the access to collected data during autonomous missions of long duration.

The present dissertation focuses on the study of mechanisms that lead to a flexibility and precision improvement of an experimental USBL system to be used in an AUV with small dimensions, intended to operate for short and long range. Firstly, it is described the architecture design of a module that is capable of improving the precision of the time of arrival measurement of signals sent by an acoustic transmitter. Then, a study is conducted on possible methods for evaluating sensor configuration performance, as it consists on a crucial factor in estimation precision. Lastly, the adaptive configuration selection method is presented, which serves as a tool that reconfigures the set of active hydrophones, from a discrete group in fixed known positions, depending on the estimated transmitter location. This method intends to achieve a higher localization precision and rectify issues that arise from classic USBL systems.

After implementation, all developed mechanism were subjected to comprehensive simulated tests that validate its function and demonstrate promising results in controlled conditions. Additionally, preliminary tests were performed in laboratory environment, however the field tests were not executed as intended due to the current pandemic situation.